\section{Experiments and Iterative Improvements}
\label{sec:experiments}
This section summarizes the iterative changes evaluated during development.
The ordering follows \texttt{report/drafts/experiments.md}, and each entry
corresponds to a submission and related code changes. I report quantitative
results in Section~\ref{sec:results}; here I focus on the intent and design
of each modification.

\subsection{Baseline}
The baseline uses LoRA adapters on attention projections, standard task
prompts, and format validation with limited retries. MCQ uses log-probability
scoring, while SAQ uses constrained generation with a strict ``Answer:''
prefix.

\subsection{SAQ Iterations}
I evaluate three categories of changes for SAQ: prompt/format refinements,
validation retries, and expanded LoRA target layers. These are designed to
reduce parsing errors, support multiword answers, and increase adapter
capacity without full fine-tuning.

\subsection{MCQ Iterations}
For MCQ, I test logprob variants and reranking weights. The goal is to
stabilize single-letter selection and leverage country priors derived from the
training set when options are ambiguous.

\subsection{RAG Variants}
RAG experiments compare raw retrieval, stop-word filtering, and stemming
configurations. I also analyze a later RAG training run and the resulting
inference behavior to assess whether retrieval helps SAQ accuracy.
