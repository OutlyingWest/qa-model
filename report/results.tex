\section{Results}
\label{sec:results}

We evaluate performance on MCQ and SAQ tasks both overall and by country,
comparing the baseline model with the best combined configuration (without RAG).

\paragraph{MCQ Performance.}
The best combined configuration increases overall MCQ accuracy from 0.74 to 0.79
(+0.05 absolute) using logprob scoring with rerank weight $w{=}2.5$. The largest
improvements are observed for CN (+0.08) and IR (+0.06), while gains for US are
moderate (+0.04) and minimal for GB (+0.01).

\paragraph{SAQ Performance.}
For SAQ, overall accuracy rises from 0.50 to 0.59 (+0.09). The largest increase
is observed for CN (+0.17), followed by US (+0.07). GB (+0.07) and IR (+0.05)
show smaller but consistent improvements.

\paragraph{RAG Configuration.}
The final RAG-based run achieves 0.58 overall SAQ accuracy, with country-level
scores of 0.51 (CN), 0.63 (GB), 0.50 (IR), and 0.70 (US). This is slightly below
the best non-RAG configuration. MCQ accuracy remains unchanged at 0.79, as RAG
is applied only to SAQ.

Overall, the combined configuration yields consistent gains across most countries,
with particularly strong improvements on SAQ for CN and US.

\begin{table}[t]
  \centering
  \begin{tabular}{lcc}
    \hline
    \textbf{MCQ} & \textbf{Baseline} & \textbf{Best} \\
    \hline
    Overall & 0.74 & 0.79 \\
    CN & 0.66 & 0.74 \\
    Iran & 0.62 & 0.68 \\
    GB & 0.90 & 0.91 \\
    US & 0.80 & 0.84 \\
    \hline
  \end{tabular}
  \caption{MCQ accuracy for the baseline and best combined configuration.}
  \label{tab:mcq}
\end{table}

\begin{table}[t]
  \centering
  \begin{tabular}{lcc}
    \hline
    \textbf{SAQ} & \textbf{Baseline} & \textbf{Best} \\
    \hline
    Overall & 0.50 & 0.59 \\
    CN & 0.40 & 0.57 \\
    GB & 0.59 & 0.66 \\
    IR & 0.38 & 0.43 \\
    US & 0.64 & 0.71 \\
    \hline
  \end{tabular}
  \caption{SAQ accuracy for the baseline and best combined configuration.}
  \label{tab:saq}
\end{table}
